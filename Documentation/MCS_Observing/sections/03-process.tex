% 03-process.tex - Process

\section{Process}

The process for defining, scheduling, and conducting observations is as follows:

\begin{enumerate}

\item The PI creates a session definition file (SDF).  This is a human-readable text file whose format is described in Section~\ref{ch:sdf-format}.  Each SDF defines exactly one session.

\item An operator submits the SDF using the task processor program \texttt{tpss}.  The following occurs:

\begin{enumerate}
\item The SDF is parsed and checked for errors, and any derived information is displayed to the operator for confirmation.

\item Resources are assigned.  For example, it is at this point that the operator might assign a particular NDP beam output to the session.  Concurrently, \texttt{tpss} is checking for resource conflicts (e.g., making sure the assigned beam is not already allocated at the requested time).  The operator also ensures that a reasonable value is assigned to \texttt{SESSION\_CRA} (see description below), and that this value is consistent with other concurrently-running sessions and the station-level CRA policy.

\item The operator edits the SDF \texttt{REMPO} fields (\texttt{PROJECT\_REMPO} and \texttt{SESSION\_REMPO}) as appropriate.  These are remarks pertaining to review and scheduling that the operator wishes to convey as metadata.  \texttt{tpss} is run at least once so that these comments are included in the version of the SDF that is submitted to MCS/Executive.

\item \texttt{tpss} outputs the following files:\footnote{In the case of \texttt{STEPPED} mode observations, the keywords associated with step definitions are not written into the modified SDF but instead are written directly into the observation specification file.}
\begin{itemize}
\item The SDF, now modified as a result of the steps above.  In this version of the SDF, the value assigned to every keyword is explicitly indicated.  The name of this file has the format \texttt{\$\{PROJECT\_ID\}\_\$\{SESSION\_ID\}.txt}.
\item A \textit{session specification file}, named \texttt{\$\{PROJECT\_ID\}\_\$\{SESSION\_ID\}.ses}.  This is a binary file that specifies parameters that apply session-wide.  These parameters are defined in Section~5.
\item One \textit{observation specification file} per observation, named \texttt{\$\{PROJECT\_ID\}\_\$\{SESSION\_ID\}\_\$\{OBS\_ID\}.obs}.  These are binary files that completely specify the scheduled observations.  The format of these files is given in Section~6.  It is these files, plus the session specification file, that are the actual input to MCS/Executive for conducting observations.
\end{itemize}
\end{enumerate}

\item The session runs at the scheduled time.  Primary NDP output data is captured by MCS-DR.

\item As each observation concludes, MCS creates a modified copy of the observation specification file with filename \texttt{\$\{PROJECT\_ID\}\_\$\{SESSION\_ID\}\_\$\{OBS\_ID\}\_\$\{OBS\_OUTCOME\}.dat}.  \texttt{\$\{OBS\_OUTCOME\}} is an integer which indicates the outcome of the observation, such as whether the observation succeeded or failed (see Section~\ref{ch:session-metadata}).  For parameters in the SDF which were set to ``MCS Decides'', the contents of this file reflect the actual values used.

\item At the conclusion of the session, MCS creates a gzipped tarball named \texttt{\$\{PROJECT\_ID\}\_\$\{SESSION\_ID\}.tgz} containing the following:

\begin{itemize}
\item The SDF, \texttt{.ses} file, and \texttt{.obs} files.
\item An \texttt{.ipl} file (``inprocessing log'').
\item A \texttt{.cs} file (``command script'').
\item A session metadata file, named \texttt{\$\{PROJECT\_ID\}\_\$\{SESSION\_ID\}\_metadata.txt} (see Section~\ref{ch:session-metadata}).
\item An \texttt{mcs.host} file containing the hostname of the MCS computer.  This is useful for determining which station produced a given metadata file.
\item A \texttt{mindelay.txt} file containing minimum beam delay information.  This is useful for correlating data from multiple stations when using the LWA Swarm as an interferometer (see~[6]).
\item Optionally, the station static MIB (\texttt{ssmif.dat}), if \texttt{SESSION\_INC\_SMIB} was set.  The format of this file is described in~[3].
\item ASP MIB snapshots taken at the beginning and end of each observation (\texttt{ASP\_begin.gdb} and \texttt{ASP\_end.gdb}).
\item A ``dynamic'' subdirectory containing \texttt{sdm.dat} (Station Dynamic MIB).  The format of this file is described in~[3].
\item If \texttt{SESSION\_LOG\_SCH}$= 1$, the MCS/Scheduler log file, \texttt{mselog.txt}, is included; it is edited to cover the time period of the session.
\item If \texttt{SESSION\_LOG\_EXE}$= 1$, the MCS/Executive log file, \texttt{meelog.txt}, is included in the ``dynamic'' subdirectory; it is edited to cover the time period of the session.
\end{itemize}

\end{enumerate}
